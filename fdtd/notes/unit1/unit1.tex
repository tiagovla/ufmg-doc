\documentclass[12pt]{article}
\usepackage{lipsum}
\usepackage{booktabs}
\usepackage{tabularx}
\usepackage{amsmath, amssymb}
\usepackage{bm}
\usepackage[acronym, section=section]{glossaries}
\usepackage{listings}
\usepackage{geometry}
\geometry{
	a4paper,
	total={170mm,257mm},
	left=20mm,
	top=20mm,
}

\title{\textbf{Notes from Unit 1 \\ Finite Difference Time Domain}}
\author{Tiago Vilela}
\date{December 26, 2020}
\newacronym{FDTD}{FDTD}{Finite Difference Time Domain}
\newacronym{PDE}{PDE}{Partial Diferential Equation}

\makeglossary

\begin{document}

\maketitle
\section{Introduction}%
\label{sec:introduction}

\acrfull{FDTD} method is a numerical analysis technique used to approximate solutions of differential equations.


\begin{table}[htpb]
	\begin{tabularx}
		{\linewidth}{>{\parskip1ex}X@{\kern4\tabcolsep}>{\parskip1ex}X}
		\toprule
		\hfil\bfseries Pros
		 &
		\hfil\bfseries Cons
		\\\cmidrule(r{3\tabcolsep}){1-1}\cmidrule(l{-\tabcolsep}){2-2}

		%% PROS, seperated by empty line or \par
		Simple and intuitive;                  \par
		Works on Field variables; \par
		Do not invert matrices;                \par
		Double precision;                      \par
		Directly models nonlinear phenomenons and several mediums. \par

		 &

		%% CONS, seperated by empty line or \par
		High processing time for 3D problems; \par
		Complex geometry demands a lot of work. \par

		\\\bottomrule
	\end{tabularx}
\end{table}

\subsection{Identification of Electromagnetic Problems}%
\label{sec:identification_of_electromagnetic_problems}
Continuous problems are caregorized based on:
\begin{itemize}
	\item the solution region;
	\item the equation that describes the problem;
	\item border conditions associated to the problem.
\end{itemize}

Electromagnetic problems can usually be classified based on the equations that describe them. These can be integral or/and differential equations. Most of them can be defined by the following equation:
\begin{equation}
	L \Phi = g
\end{equation}
where L is a differential, integral, or integral-diferential operator, g is the source (usually known) and $\Phi$ is the function to be solved.

\begin{equation}
	a\frac{\partial^{2}\Phi}{\partial x^{2}}+
	b\frac{\partial^{2}\Phi}{\partial x\partial y}+
	c\frac{\partial^{2}\Phi}{\partial y^{2}}+
	d\frac{\partial\Phi}{\partial x}+
	e\frac{\partial\Phi}{\partial y}+
	f\Phi=
	g
\end{equation}

\begin{equation}
	L=
	a\frac{\partial^{2}}{\partial x^{2}}+
	b\frac{\partial^{2}}{\partial x \partial y}+
	c\frac{\partial^{2}}{\partial y^{2}}+
	d\frac{\partial}{\partial x}+
	e\frac{\partial}{\partial y}+
	f
\end{equation}
where the coeficients a,b,c are functions of $x$ and $y$, and and might be dependent on $\Phi$, in case the \acrfull{PDE} is nonlinear. Additionally, the \acrshort{PDE} is homogeneous if $g(x,y) = 0$ and non homogeneous if $g(x,y) \neq 0$.

\subsection{Second Order Differential Equations}%
\label{sub:second_order_differential_equations}


Derived from the equation $a x² + b xy+ c y^{2} + d x+ e y+ f = 0$, the second order \acrshortpl{PDE} are classified by:
\begin{table}[htpb]
	\centering
	\label{tab:second_order_pde_classification}
	\begin{tabular}{cc}
		\toprule
		condition  & case       \\
		\midrule
		$b²-4ac<0$ & eliptic    \\
		$b²-4ac>0$ & hyperbolic \\
		$b²-4ac=0$ & parabolic  \\
		\bottomrule
	\end{tabular}
\end{table}


\begin{table}[htpb]
	\centering
	\caption{Classification of Electromagnetic Problems}
	\label{tab:label}
	\begin{tabular}{cccc}
		\toprule
		Case       & Equation                                                                                             & Variables                   & Associated to           \\
		\midrule
		Eliptic    & $\frac{\partial^{2}\Phi}{\partial x^{2}} + \frac{\partial^{2}\Phi}{\partial y^{2}} = g(x,y) $        & $a=c=1$ and $b=0$           & Stationary  phenomenons \\
		Hyperbolic & $\frac{\partial^{2}\Phi}{\partial x^{2}} = \frac{1}{u^{2}} \frac{\partial^{2}\Phi}{\partial t^{2}} $ & $a=u^{2}$, $b=0$ and $c=-1$ & Propagation problems    \\
		Parabolic  & $\frac{\partial^{2}\Phi}{\partial x^{2}} =k \frac{\partial\Phi}{\partial t}$                         & $a=1$ and $b=c=0$           & Slow change phenomenons \\
		\bottomrule
	\end{tabular}
\end{table}

The solution $\Phi$ in a bounded region  $R$ is required to satisfy given conditions  $S$, these are:

\begin{table}[htpb]
	\centering
	\caption{Boundary Conditions}
	\label{tab:boundary_conditions}
	\begin{tabular}{ccc}
		\toprule
		Condition           & Equation                                                                   \\
		\midrule
		Dirichlet Condition & $\Phi(\bm{r}) = 0$                                                         \\
		Neumann Condition   & $\frac{\partial \Phi(\bm{r})}{\partial n} = 0$                             \\
		Neumann Condition   & $\frac{\partial \Phi(\bm{r})}{\partial n} +  \Phi(\bm{r}) + h(\bm{r}) = 0$ \\
		\bottomrule
	\end{tabular}
\end{table}

\end{document}

